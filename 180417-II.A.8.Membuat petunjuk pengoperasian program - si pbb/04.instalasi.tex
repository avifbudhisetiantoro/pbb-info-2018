\chapter{INSTALASI PROGRAM}

Agar aplikasi atau sistem informasi berjalan dengan semestinya, ada beberapa perangkat lunak yang diperlukan seperti berikut :

\begin{itemize}
	\item Java Development Kit 1.8 atau yang lebih baru
	
Ini digunakan untuk menjalankan \textit{service} atau layanan yang berada pada ujung-belakang (\textit{backend}).	
	
	\item \textit{Web Server} (dalam hal ini menggunakan Nginx web server)
	
Ini digunakan untuk melayani aplikasi \textit{web} pada bagian ujung-depan (\textit{frontend}), jadi sesungguhnya, saat awal pengguna terhubung dengan sistem informasi pembayaran PBB-P2 ini, mereka akan melakukan akses ke \textit{web server} ini, setelah mendapatkan halaman dari aplikasi \textit{web}, aplikasi \textit{web} inilah yang akan melakukan komunikasi dengan \textit{service} atau layanan yang berada pada ujung-belakang (\textit{backend}).
\end{itemize}

Langkah instalasi atau pemasangannya cukup mudah, hanya tinggal \textit{copy-paste} (salin-tempel) aplikasi bagian-belakang (\textit{backend}) yang memiliki ekstensi \texttt{jar} ke peladen, kemudian jalankan dengan perintah berikut :

\begin{lstlisting}[language=java]
java -jar e-pbb.jar &
\end{lstlisting}

Aplikasi ini memiliki sebuah \textit{servlet container} yang terintegrasi berupa Apache Tomcat karena dibangun menggunakan Springboot, sehingga apabila dijalankan atau dieksekusi langsung seperti di atas, \textit{servlet container} akan otomatis dijalankan untuk melayani pengguna (dalam hal ini adalah aplikasi bagian-depan / \textit{frontend}).

Untuk instalasi atau pemasangan aplikasi bagian-depan (\textit{frontend}), karena dibangun menggunakan Angular 5, maka hasil \textit{build} dari kode yang telah dibangun adalah seperti berikut :

\begin{lstlisting}[language=sh]
favicon.ico
index.html
inline.bundle.js
inline.bundle.js.map
main.bundle.js
main.bundle.js.map
polyfills.bundle.js
polyfills.bundle.js.map
styles.bundle.js
styles.bundle.js.map
vendor.bundle.js
vendor.bundle.js.map
\end{lstlisting}

Seluruh \textit{file} / berkas tersebut dipindahkan ke kandar / direktori \textit{root} dari \textit{web server}. Sampai langkah ini aplikasi baik bagian-depan (\textit{frontend}) maupun bagian-belakang (\textit{backend}) sudah siap untuk melayani.