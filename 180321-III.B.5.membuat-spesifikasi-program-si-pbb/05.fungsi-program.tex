\chapter{FUNGSI-FUNGSI YANG HARUS DILAKUKAN OLEH PROGRAM}

Fungsi yang harus dilakukan oleh program atau sistem informasi yang akan dibangun tentunya harus dapat menampilkan beberapa informasi seperti berikut ini :

\begin{itemize}
	\item Data objek pajak sebagai bahan konfirmasi dan verifikasi bahwa data dengan Nomor Objek Pajak (NOP) yang diinginkan oleh pengguna benar, termasuk di dalamnya adalah Nomor Objek Pajak (NOP), luas bumi dan bangunan, Nilai Jual Objek Pajak Bumi dan Bangunan seperti tertera dalam lembar Surat Pemberitahuan Pajak Terhutang (SPPT), serta lokasi objek pajak berada.
	\item Data subjek pajak, ini pun sebagai bahan konfirmasi dan verifikasi bahwa data subjek pajak yang nantinya ditetapkan sebagai wajib pajak adalah benar seperti tercantum dalam lembar Surat Pemberitahuan Pajak Terhutang (SPPT). Termasuk di dalamnya adalah data-data seperti nomor identitas subjek pajak, nama subjek pajak, serta alamat tempat tinggal subjek pajak.
	\item Data tagihan dari Surat Pemberitahuan Pajak Terhutang (SPPT) yang ditampilkan secara rinci untuk tiap tahun pajak. Termasuk di dalamnya adalah informasi mengenai tahun pajak, besarnya tagihan terhutang, dan status pembayaran yang tercatat.
\end{itemize}